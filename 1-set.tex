\documentclass[12pt]{article}



\usepackage{amsmath, amsthm, amssymb, amsfonts}
\usepackage{mathtools}
\usepackage{thmtools}
\usepackage{etoolbox}
\usepackage{enumerate, enumitem}
\usepackage{tikz}
\usepackage[dvipsnames]{xcolor}
\usepackage[framemethod=TikZ]{mdframed}


\def\theenumi{(\arabic{enumi})}
\def\thefootnote{(\arabic{footnote})}



\def\R{\mathbb{R}}
\def\Q{\mathbb{Q}}
\def\Z{\mathbb{Z}}
\def\N{\mathbb{N}}

% ==> Definition
\mdfdefinestyle{mdDefinitionStyle}{
  % roundcorner       = 0pt,
  % linewidth         = 0.8pt,
  skipabove         = 12pt,
  innerbottommargin = 9pt,
  skipbelow         = 2pt,
  nobreak           = false,
  % linecolor         = Gray,
  % backgroundcolor   = TealBlue!8,
}
\declaretheoremstyle[
  headfont      = {\sffamily\bfseries},
  bodyfont      = {\normalfont},
  mdframed      = {style=mdDefinitionStyle},
  headpunct     = {\\[3pt]},
  postheadspace = {0pt},
  notefont      = {\normalfont\sffamily\small},
  notebraces    = {~(}{)},
]{mdDefinition}

% ==> Theorem
\mdfdefinestyle{mdTheoremStyle}{%
  % linewidth           = 1pt,
  skipabove           = 12pt,
  frametitleaboveskip = 5pt,
  frametitlebelowskip = 0pt,
  skipbelow           = 2pt,
  innertopmargin      = 5pt,
  innerbottommargin   = 8pt,
  nobreak             = false,
  % linecolor           = Gray,
  % backgroundcolor     = Apricot!10,
}
\declaretheoremstyle[
  headfont      =\bfseries\sffamily,
  headindent    =0pt,
  bodyfont      =\itshape,
  mdframed      ={style=mdTheoremStyle},
  notefont      =\normalfont\sffamily\small,
  notebraces    = {~(}{)},
  headpunct     ={\(:\) },
  postheadspace ={2pt},
]{mdTheorem}



\declaretheorem[style=mdDefinition, name={Definition}, ]{definition}
\declaretheorem[style=mdTheorem, name={Theorem}]{theorem}
% \declaretheorem{example}[name={\exampleName},style=mdExample, numberwithin=chapter]
% \declaretheorem{exercise}[name={\exerciseName},style=mdExercise, numberwithin=chapter]

\theoremstyle{definition}
\newtheorem{exercise}{Exercice}



\renewcommand\epsilon{\varepsilon}
\def\R{\mathbb{R}}
\def\Q{\mathbb{Q}}
\def\Z{\mathbb{Z}}
\def\N{\mathbb{N}}
\mathtoolsset{centercolon} % not work when using |mathpazo|
\DeclarePairedDelimiter\abs{\lvert}{\rvert}
\DeclarePairedDelimiter\floor{\lfloor}{\rfloor}
\DeclarePairedDelimiterX\norm[1]\lVert\rVert{
	\ifblank{#1}{\:\cdot\:}{#1}
}



\usepackage{geometry} %[showframe]
\geometry{
  top=25.4mm,         % 1 inch
  bottom=41mm,        % 1.618 inches
  left=38mm,          % 1.5 inches
  right=38mm,         % 1.5 inches
}

% \geometry{
%   top=25.4mm,         % 1 inch
%   bottom=41mm,        % Golden ratio for bottom margin
%   inner=25.4mm,       % Smaller inner margin (1 inch)
%   outer=41mm,         % Larger outer margin (Golden ratio: 1.618 inches)
%   bindingoffset=10mm  % Optional space for binding (adjust as needed)
% }



% \newcounter{tdcounter}
\newcommand{\tdtitle}[2]{
  \noindent\rule{\textwidth}{0.8pt}
  \flushleft
    {\slshape Royal University of Phnom Penh \hfill Elementary Real Analysis}\\
    {\slshape Undergraduate Mathematics \hfill Class of 2024-2025}
  \endflushleft
  
  \vspace{0.8mm}
  \center
  {\sffamily\large TD n\textdegree #1 ~\textendash~ #2}
  \endcenter
  \noindent\rule{\textwidth}{0.8pt}
}



% \renewcommand{\qedsymbol}{\(\blacksquare\)}
\makeatletter
\renewenvironment{proof}[1][\proofname]{\par
  \pushQED{\qed}%
  \normalfont \topsep6\p@\@plus6\p@\relax
  \trivlist
\item\relax
  {\sffamily #1\@addpunct{.}}\hspace\labelsep\ignorespaces
}{%
  \popQED\endtrivlist\@endpefalse
}
\makeatother

\title{Elementary Real Analysis}
\author{\textsc{Sivmeng HUN}}
\date{32 Novembre 2024}

\usepackage{hyperref}


\begin{document}
\maketitle

\section{Completeness Property}

\subsection{Relation between \(\N\) and \(\R\)}


Here we assume that there exists \(\N\) a subset of \(\R\) with the following
properties:
\begin{enumerate}[label = {(\roman*).}]
\item \(1\in\N\), and is the smallest element in \(\N\);
\item If \(n\in\N\), then \(n+1\in\N\);
\item If \(n,m\in\N\) such that \(n\neq m\), then
  \(\abs{n-m}\geq 1\).
\end{enumerate}

\begin{theorem}[Archimedean Property]
  The set \(\N\) is not bounded above. In other words,
  for any \(x\in\R\), there exists \(n\in\N\) such that \(n>x\). 
\end{theorem}
\begin{proof}
  Assume by contradiction that the above statement is wrong, that is
  the set \(\N\) is bounded above.
  Thus \(\alpha = \sup\N\) exists. By definition of supremum,
  there is \(n\in\N\) so that
  \[
    n > \alpha - 1 \implies \alpha < n+1.
  \]
  However, this is a contradiction since \(n+1\in\N\) and \(\alpha\)
  is supposed to be an upper bound of \(\N\).
\end{proof}

\begin{theorem}[Well-ordering property]
  Any non-empty subset \(S\subset\N\) has a minimal element;
  in other words \(\min S\in S\).
\end{theorem}
\begin{proof}
  Let \(S\subset\N\) and \(S\neq\varnothing\). Since \(\N\) bounded below,
  then so does \(S\). We conclude from completeness property that
  \(\alpha = \inf S\in\R\) exists.
  It suffices to prove that \(\alpha \in S\). Argue by contradiction and
  suppose that \(\alpha\notin S\). In particular, \(\alpha\) is an a natural
  number.
  %
  From definition of infimum, there is an element \(s\in S\) so that
  \(\alpha \leq s<\alpha+1  \).
  Moreover we cannot have \(\alpha=s\), since we assumed \(\alpha\notin S\).
  Thus we have found \(s\in S\) such that
  \[\alpha<s<\alpha+1.\]
  Notice that \(s\) is a number that is greater than \(\alpha=\inf S\).
  Using definition of infimum again, we conclude that there is
  \(s'\in S\) with \(\alpha \leq s' < s\). Using the same argument as above,
  we must have \(\alpha < s'\). Combining all inequalities, we obtain
  \[\alpha < s' < s < \alpha+1. \]
  Hence \(0< s-s' <  1\). This is a contradiction to the (iii) property
  of \(\N\).
\end{proof}



\subsection{Relation between \(\Q\) and \(\R\)}

\begin{theorem}[Density of \(\Q\) in \(\R\)]
  For any two distinct real numbers \(x,y\in\R\) with \(x<y\),
  there is a rational number \(r\in\Q\) satisfying
  \(x<r<y\).
\end{theorem}
We say that a set \(A\subset\R\) is dense provided that
for any \(x<y\), there exists \(a\in A\) with \(x<a<y\).
From the above theorem, the set \(\Q\) has this exact property.
In other words, this means that no matter how you choose \(x\) and \(y\),
there is always an element in \(\Q\) sits between them.
Visually, the elements of \(\Q\) are densely put into \(\R\),
that is why we say \(\Q\) is dense in \(\R\).

The machinery of the proof is to use Archimedean property.
To shorten the proof a bit, we are going to use the fact that,
for any two numbers that are strictly of distance 1 apart,
there is an integer sits between them. This fact is left as
an exercise, as you can see in
\href{https://sivmeng.com/real-analysis-2425/td/td1.pdf}{\sffamily TD n\textdegree 1}.

\begin{proof}
  Let \(x<y\) be real numbers. From Archimedean property,
  there is an natural number \(n\in\N\) such that \(n>\frac{1}{y-x}\).
  Equivalently,
  \[ny - nx >1.\]
  Since \(ny\) and \(nx\) are strictly of distance 1 part,
  there is an integer \(m\in\Z\) such that \(nx<m<ny\).
  Thus
  \[x < \frac{m}{n} < y.\]
  Therefore, we have found a rational number \(\frac{m}{n}\)
  that is between \(x\) and \(y\). This concludes the proof.
\end{proof}






\end{document}
