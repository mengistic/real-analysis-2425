

\usepackage{amsmath, amsthm, amssymb, amsfonts}
\usepackage{mathtools}
\usepackage{thmtools}
\usepackage{etoolbox}
\usepackage{enumerate, enumitem}
\usepackage{tikz}
\usepackage[dvipsnames]{xcolor}
\usepackage[framemethod=TikZ]{mdframed}


\def\theenumi{(\arabic{enumi})}
\def\thefootnote{(\arabic{footnote})}



\def\R{\mathbb{R}}
\def\Q{\mathbb{Q}}
\def\Z{\mathbb{Z}}
\def\N{\mathbb{N}}

% ==> Definition
\mdfdefinestyle{mdDefinitionStyle}{
  % roundcorner       = 0pt,
  % linewidth         = 0.8pt,
  skipabove         = 12pt,
  innerbottommargin = 9pt,
  skipbelow         = 2pt,
  nobreak           = false,
  % linecolor         = Gray,
  % backgroundcolor   = TealBlue!8,
}
\declaretheoremstyle[
  headfont      = {\sffamily\bfseries},
  bodyfont      = {\normalfont},
  mdframed      = {style=mdDefinitionStyle},
  headpunct     = {\\[3pt]},
  postheadspace = {0pt},
  notefont      = {\normalfont\sffamily\small},
  notebraces    = {~(}{)},
]{mdDefinition}

% ==> Theorem
\mdfdefinestyle{mdTheoremStyle}{%
  % linewidth           = 1pt,
  skipabove           = 12pt,
  frametitleaboveskip = 5pt,
  frametitlebelowskip = 0pt,
  skipbelow           = 2pt,
  innertopmargin      = 5pt,
  innerbottommargin   = 8pt,
  nobreak             = false,
  % linecolor           = Gray,
  % backgroundcolor     = Apricot!10,
}
\declaretheoremstyle[
  headfont      =\bfseries\sffamily,
  headindent    =0pt,
  bodyfont      =\itshape,
  mdframed      ={style=mdTheoremStyle},
  notefont      =\normalfont\sffamily\small,
  notebraces    = {~(}{)},
  headpunct     ={\(:\) },
  postheadspace ={2pt},
]{mdTheorem}



\declaretheorem[style=mdDefinition, name={Definition}, ]{definition}
\declaretheorem[style=mdTheorem, name={Theorem}]{theorem}
% \declaretheorem{example}[name={\exampleName},style=mdExample, numberwithin=chapter]
% \declaretheorem{exercise}[name={\exerciseName},style=mdExercise, numberwithin=chapter]

\theoremstyle{definition}
\newtheorem{exercise}{Exercice}



\renewcommand\epsilon{\varepsilon}
\def\R{\mathbb{R}}
\def\Q{\mathbb{Q}}
\def\Z{\mathbb{Z}}
\def\N{\mathbb{N}}
\mathtoolsset{centercolon} % not work when using |mathpazo|
\DeclarePairedDelimiter\abs{\lvert}{\rvert}
\DeclarePairedDelimiter\floor{\lfloor}{\rfloor}
\DeclarePairedDelimiterX\norm[1]\lVert\rVert{
	\ifblank{#1}{\:\cdot\:}{#1}
}



\usepackage{geometry} %[showframe]
\geometry{
  top=25.4mm,         % 1 inch
  bottom=41mm,        % 1.618 inches
  left=38mm,          % 1.5 inches
  right=38mm,         % 1.5 inches
}

% \geometry{
%   top=25.4mm,         % 1 inch
%   bottom=41mm,        % Golden ratio for bottom margin
%   inner=25.4mm,       % Smaller inner margin (1 inch)
%   outer=41mm,         % Larger outer margin (Golden ratio: 1.618 inches)
%   bindingoffset=10mm  % Optional space for binding (adjust as needed)
% }



% \newcounter{tdcounter}
\newcommand{\tdtitle}[2]{
  \noindent\rule{\textwidth}{0.8pt}
  \flushleft
    {\slshape Royal University of Phnom Penh \hfill Elementary Real Analysis}\\
    {\slshape Undergraduate Mathematics \hfill Class of 2024-2025}
  \endflushleft
  
  \vspace{0.8mm}
  \center
  {\sffamily\large TD n\textdegree #1 ~\textendash~ #2}
  \endcenter
  \noindent\rule{\textwidth}{0.8pt}
}



% \renewcommand{\qedsymbol}{\(\blacksquare\)}
\makeatletter
\renewenvironment{proof}[1][\proofname]{\par
  \pushQED{\qed}%
  \normalfont \topsep6\p@\@plus6\p@\relax
  \trivlist
\item\relax
  {\sffamily #1\@addpunct{.}}\hspace\labelsep\ignorespaces
}{%
  \popQED\endtrivlist\@endpefalse
}
\makeatother
