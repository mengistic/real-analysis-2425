\documentclass[10pt, a4paper]{article}


\usepackage{amsmath, amsthm, amssymb, amsfonts}
\usepackage{mathtools}
\usepackage{thmtools}
\usepackage{etoolbox}
\usepackage{enumerate, enumitem}
\usepackage{tikz}
\usepackage[dvipsnames]{xcolor}
\usepackage[framemethod=TikZ]{mdframed}


\def\theenumi{(\arabic{enumi})}
\def\thefootnote{(\arabic{footnote})}



\def\R{\mathbb{R}}
\def\Q{\mathbb{Q}}
\def\Z{\mathbb{Z}}
\def\N{\mathbb{N}}

% ==> Definition
\mdfdefinestyle{mdDefinitionStyle}{
  % roundcorner       = 0pt,
  % linewidth         = 0.8pt,
  skipabove         = 12pt,
  innerbottommargin = 9pt,
  skipbelow         = 2pt,
  nobreak           = false,
  % linecolor         = Gray,
  % backgroundcolor   = TealBlue!8,
}
\declaretheoremstyle[
  headfont      = {\sffamily\bfseries},
  bodyfont      = {\normalfont},
  mdframed      = {style=mdDefinitionStyle},
  headpunct     = {\\[3pt]},
  postheadspace = {0pt},
  notefont      = {\normalfont\sffamily\small},
  notebraces    = {~(}{)},
]{mdDefinition}

% ==> Theorem
\mdfdefinestyle{mdTheoremStyle}{%
  % linewidth           = 1pt,
  skipabove           = 12pt,
  frametitleaboveskip = 5pt,
  frametitlebelowskip = 0pt,
  skipbelow           = 2pt,
  innertopmargin      = 5pt,
  innerbottommargin   = 8pt,
  nobreak             = false,
  % linecolor           = Gray,
  % backgroundcolor     = Apricot!10,
}
\declaretheoremstyle[
  headfont      =\bfseries\sffamily,
  headindent    =0pt,
  bodyfont      =\itshape,
  mdframed      ={style=mdTheoremStyle},
  notefont      =\normalfont\sffamily\small,
  notebraces    = {~(}{)},
  headpunct     ={\(:\) },
  postheadspace ={2pt},
]{mdTheorem}



\declaretheorem[style=mdDefinition, name={Definition}, ]{definition}
\declaretheorem[style=mdTheorem, name={Theorem}]{theorem}
% \declaretheorem{example}[name={\exampleName},style=mdExample, numberwithin=chapter]
% \declaretheorem{exercise}[name={\exerciseName},style=mdExercise, numberwithin=chapter]

\theoremstyle{definition}
\newtheorem{exercise}{Exercice}



\renewcommand\epsilon{\varepsilon}
\def\R{\mathbb{R}}
\def\Q{\mathbb{Q}}
\def\Z{\mathbb{Z}}
\def\N{\mathbb{N}}
\mathtoolsset{centercolon} % not work when using |mathpazo|
\DeclarePairedDelimiter\abs{\lvert}{\rvert}
\DeclarePairedDelimiter\floor{\lfloor}{\rfloor}
\DeclarePairedDelimiterX\norm[1]\lVert\rVert{
	\ifblank{#1}{\:\cdot\:}{#1}
}



\usepackage{geometry} %[showframe]
\geometry{
  top=25.4mm,         % 1 inch
  bottom=41mm,        % 1.618 inches
  left=38mm,          % 1.5 inches
  right=38mm,         % 1.5 inches
}

% \geometry{
%   top=25.4mm,         % 1 inch
%   bottom=41mm,        % Golden ratio for bottom margin
%   inner=25.4mm,       % Smaller inner margin (1 inch)
%   outer=41mm,         % Larger outer margin (Golden ratio: 1.618 inches)
%   bindingoffset=10mm  % Optional space for binding (adjust as needed)
% }



% \newcounter{tdcounter}
\newcommand{\tdtitle}[2]{
  \noindent\rule{\textwidth}{0.8pt}
  \flushleft
    {\slshape Royal University of Phnom Penh \hfill Elementary Real Analysis}\\
    {\slshape Undergraduate Mathematics \hfill Class of 2024-2025}
  \endflushleft
  
  \vspace{0.8mm}
  \center
  {\sffamily\large TD n\textdegree #1 ~\textendash~ #2}
  \endcenter
  \noindent\rule{\textwidth}{0.8pt}
}



% \renewcommand{\qedsymbol}{\(\blacksquare\)}
\makeatletter
\renewenvironment{proof}[1][\proofname]{\par
  \pushQED{\qed}%
  \normalfont \topsep6\p@\@plus6\p@\relax
  \trivlist
\item\relax
  {\sffamily #1\@addpunct{.}}\hspace\labelsep\ignorespaces
}{%
  \popQED\endtrivlist\@endpefalse
}
\makeatother



\begin{document}


%%% TD 1 
%%%%%%%%%%%%%%%%%%%%%%%%%%%%%%%%%%%%%%%%%%%%%%%%%%
{\tdtitle{1}{Set of real numbers}}

\begin{exercise}
  We say the set \(A\subset\R\) is \emph{bounded} provided \(A\) is
  both bounded above and bounded below.
  Prove that %\(A\) is bounded \emph{if and only if} \(\exists M>0\) such that \(\forall a\in A,~ \abs{a}\leq M\).
  \(
    \text{A is bounded}\iff
    \exists M>0,~\forall a\in A,~\abs{a}\leq M.
  \)
\end{exercise}

\begin{exercise}
  Determine whether the following sets are bounded below, or bounded above.
  (\textsf{Note:} You are allowed to use \(\floor{\cdot}\) in this exercise.)
  \begin{enumerate}
  \item \(A = \{\frac{n}{2}+1 \colon n\in \N\}\)
  \item \(B = \{\frac{n}{2}+1 \colon n\in \Z\}\)
  \item \(C = \{\frac{1}{n} \colon n\in\N\}\)
  \item \(D = \{ (-1)^n n \colon n\in\N\}\)
  \item \(E = \{ x\in\R \colon x^2\leq 3\}\)
    (\textsf{Indication:}
    For the last exercise even though it is tempting to use the
    \(\sqrt{\cdot}\) in order to solve for \(x\), you are advised \emph{not}
    to use it. This is because up until now we haven't showed that the
    \emph{number} \(\sqrt{3}\) exist or not.
    )
  \end{enumerate}
\end{exercise}

\begin{exercise}[Finding supremum and infimum]
  (\textsf{Note:} You are allowed to use \(\floor{\cdot}\) in this exercise.)
  \begin{enumerate}
  \item Let \(A = (-\infty, 5]\). Prove that \(\sup A = 5\).
  \item Let \(B = (2, \infty)\). Prove that \(\inf B = 2\).
  \item Let \(C = (1, 3]\). Prove that \(\sup C = 3\) and \(\inf C=1\).
  \item Let \(D = \{\frac{1}{n}\colon n\in\N\}\). Prove that \(\sup D=1\) and \(\inf D=0\).
  \end{enumerate}
\end{exercise}

\begin{exercise}
  Let \(x\in\R\) be a real number. Prove that there exists an integer
  \(n\in\N\) such that \(nx > 1\).
\end{exercise}
\begin{exercise}
  Let \(x,y\in\R\) be real numbers satisfying \(x<y\).
  Prove that there exists an integer \(n\in\N\) such that
  \(x + \frac1n < y.\)
\end{exercise}

\begin{exercise}[Existence of roots]
  \text{}
  \begin{enumerate}
  \item
    Let \(A = \{x\in\R \colon x^2 < 3\}\). Prove that \(A\) is bounded above and
    \((\sup A)^2 = 3\).
  \item 
    Let \(n\in\N\) and \(a\in\R\) with \(a > 0\).
    We denote the set  \(A = \{x\in\R \colon x^n < a\}\).
    Prove that \(A\) is bounded above and \((\sup A)^n = a\).
  \end{enumerate}
\end{exercise}

\begin{exercise}[Infimum]
  The goal of this exercise is to prove that any subset of
  \(\R\) that is bounded below has an infimum.
  Let \(A\subset\R\) be a subset that is bounded below.
  We denote
  \[B = \{-a~ \colon a\in A\}.\]
  \begin{enumerate}
  \item Prove that \(B\) is bounded above. Let \(\beta = \sup B\).
  \item Prove that \(\inf A = -\beta\). (Thus proving that infimum
    of any set that is bounded below exists.)
  \end{enumerate}
\end{exercise}

\begin{exercise}[Existence of integer part function]
  Let \(x\in\R\). Prove that there exists a unique integer \(N\in\Z\)
  satisfying \(N\leq x < N+1\). We denote this integer by \(N=\floor{x}\).
\end{exercise}

\begin{exercise}
  Let \(x,y\in\R\) with \(y-x>1\). Prove that there exists an integer
  \(m\in\Z\) such that \(x<m<y\).
\end{exercise}

\begin{exercise}
  Prove that the set \(A = \{\frac{p}{2^n}\colon p\in\Z,~n\in\N\}\)
  is dense in \(\R\).
\end{exercise}

\begin{exercise}
  Prove that the set of irrational numbers \(\R\smallsetminus\Q\)
  is dense in \(\R\).
\end{exercise}


%%% Solution
%%%%%%%%%%%%%%%%%%%%%%%%%%%%%%%%%%%%%%%%%%%%%%%%%%

% \newpage
% \setcounter{exercise}{0}


% \begin{center}
%   {\large\sffamily Solution}
% \end{center}

% \begin{exercise}
%   We say the set \(A\subset\R\) is \emph{bounded} provided \(A\) is
%   both bounded above and bounded below.
%   Prove that 
%   \(
%     \text{A is bounded}\iff
%     \exists M>0,~\forall a\in A,~\abs{a}\leq M.
%   \)
% \end{exercise}
% \begin{proof}
%   todo.
% \end{proof}


% \begin{exercise}
%   Determine whether the following sets are bounded below, or bounded above.
%   \begin{enumerate}
%   \item \(A = \{\frac{n}{2}+1 \colon n\in \N\}\)
%   \item \(B = \{\frac{n}{2}+1 \colon n\in \Z\}\)
%   \item \(C = \{\frac{1}{n} \colon n\in\N\}\)
%   \item \(D = \{ (-1)^n n \colon n\in\N\}\)
%   \item \(E = \{ x\in\R \colon x^2\leq 3\}\)
%   \end{enumerate}
% \end{exercise}
% \begin{proof}
%   todo
% \end{proof}




\end{document}
